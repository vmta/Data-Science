\documentclass[12pt,a4paper]{article}
\usepackage[utf8]{inputenc}
\usepackage[russian]{babel}
\usepackage{url}
\usepackage{amsmath}
\usepackage{amsfonts}
\usepackage{amssymb}
\usepackage[left=2cm,right=2cm,top=2cm,bottom=2cm]{geometry}
\newcommand{\e}{\varepsilon}
\renewcommand{\b}{\beta}
\newcommand{\hb}{\hat{\b}}
\newcommand{\hs}{\hat{\sigma}}
\usepackage{graphicx}
\begin{document}

5-1-1 \url{http://youtu.be/qse3g7L_8D4} Гомоскедастичность

1:18 старый заголовок <<Условная гомоскедастичность>> нужно убрать, т.е. оставить на слайде только четыре надписи с буллетами (условная гомо..., условная гетеро..., безусловная гомо..., безусловная гетеро...) без заголовка.

5-1-2 Условная гетероскедастичность 

\url{http://www.youtube.com/watch?v=E4P57J6Dp8M}   ok 

5-1-3  Безусловная гетероскедастичность 

\url{http://www.youtube.com/watch?v=znxPMtkEJaI}  ok

5-1-4 Последствия гетероскедастичности для малых выборок 

\url{http://youtu.be/QxXqvtoCt7o} ok

5-1-5 Асимптотические последствия условной гетероскедастичности

\url{http://youtu.be/jUCdLWa4_60} ок

5-1-6 Обнаружение гетероскедастичности и стандартные ошибки Уайта

\url{http://youtu.be/WlXXomMuKAk}

5:23 исправить заголовок слайда на <<С практической точки зрения:>>

5:49 исправить заголовок слайда на <<Когда следует использовать робастные оценки дисперсии?>>

10:12 начать новую строку после <<Если верна>> и \textit{исправить ошибку в формуле в конце слайда (!)}, т.е. должно получится:

Тест Уайта (заголовок слайда, синим)


* Если верна

$H_0$: условная гомоскедастичность, 

$Var(\e_i|X)=\sigma^2$

* То асимптотически $LM\sim \chi^2_{m-1}$

$m$ --- число коэффициентов 

во вспомогательной регрессии:

$\hat{\varepsilon}^2_i = \gamma_1 + \gamma_2 z_{i2} + \ldots + \gamma_{m} z_{im}+ u_i$

5 1 7 Пример. Тест Уайта

\url{http://youtu.be/sNnqO0hjU5g} ок

5 1 8 \url{http://youtu.be/sDQrtXEqhr4}
%\url{http://youtu.be/8MjWGAfyHec}
% \url{http://youtu.be/gTJ-GFTAVbk} (без переснятого куска)

3:26 немного аккуратнее оформить (добавить новую строку после <<Если верна>>, сделать одинаковое оформление у пунктов про $n_1$  и $n_2$) чтобы вышло:

* Если верна

$H_0$: условная гомоскедастичность, 

$Var(\e_i|X)=\sigma^2$

* То $F=\frac{RSS_1/(n_1-k)}{RSS_2/(n_2-k)} \sim F_{n_1-k,n_2-k}$

$n_1$ --- число наблюдений 

в <<верхней>> части выборки

$n_2$ --- число наблюдений 

в <<нижней>> части выборки

5-1-9 

\url{http://youtu.be/nfsYNHrI3KI} ок 

5 1 10 Мораль лекции о гетероскедастичности, \url{http://youtu.be/ElOJt8YyVcA}

0:16 добавить название лекции (синим по центру) <<Гетероскедастичность>>, добавить название фрагмента (на синей полосе внизу) <<Мораль лекции о гетероскедастичности>>

0:24 во втором пункте списка пропущено слово <<выборке>>, должно быть:

* Гетероскедастичность в случайной выборке почти всегда есть

5-2-1 Написание функций в R

\url{http://youtu.be/aqsxmvhiTLU} ок

5-2-2 Написание циклов в R

\url{http://youtu.be/efsml1jzQ1I} ok

5 2 3 Оценка модели с помощью мнк 

\url{http://youtu.be/S377yzJkQXg} ок

5-2-4 Доверительные интервалы при гетероскедастичности 

\url{http://youtu.be/GTcr3nY05N8} ок

5 2 5 Тесты на гетероскедастичность 

\url{http://youtu.be/vEyRr-VkTPw} 

видео меня и звук --- верные.

однако наложен скринкаст из отбракованного дубля. Надо наложить тот скринкаст, который соответствует съемке.

Мне трудно указать точное соответствие во времени, но скринкаст момента времени 5:40 примерно должен приходится примерно на момент времени 2:30. То есть скринкаст (и только скринкаст!) от начала и до 2:40 (примерно, я плохо понимаю, где склейка) надо отрезать и выкинуть.













\end{document}