\documentclass[12pt,a4paper]{article}
\usepackage[utf8]{inputenc}
\usepackage[russian]{babel}
%\usepackage[OT1]{fontenc}
\usepackage{url}
\usepackage{amsmath}
\usepackage{amsfonts}
\usepackage{amssymb}
\usepackage[left=2cm,right=2cm,top=2cm,bottom=2cm]{geometry}
\begin{document}

4-1-1 Исправить название фрагмента  на <<Определение мультиколлинеарности>> (сейчас определие)

\url{http://youtu.be/9MhZ0UCb_Cc}

0:08 убрать слова <<кафедра публичной политики>>  

0:52 заменить появляющуюся матрицу на более длинную:

\[
\begin{pmatrix}
1 & 4 & 12 & 8 \\
1 & 3 & 3   & 3 \\
1 & 1 & 7   & 4 \\
1 & 2 & 4   & 3 \\
1 & 3 & 5   & 4 \\ 
\vdots & \vdots & \vdots & \vdots
\end{pmatrix}
\]

3:53 убрать полностью <<Практика>> и два подпункта

6:07 ошибка в начале формулы (пропущен квадрат), должно быть:

$se^2(\hat{\beta}_j)=\frac{\hat{\sigma}^2}{RSS_j}=\frac{\hat{\sigma}^2}{TSS_j\cdot (1-R^2_j)}=
\frac{1}{1-R^2_j}\frac{\hat{\sigma}^2}{TSS_j}$

9:57 ошибка в начале формулы (пропущен квадрат), должно быть:

$se^2(\hat{\beta}_j)=VIF_j \cdot \frac{\hat{\sigma}^2}{TSS_j}$

10:19 после пункта <<Выборочные корреляции...>> добавить формулу

$sCorr(x,z)=\frac{\sum (x_i - \bar{x}) (z_i - \bar{z}) }{\sqrt{sVar(x) \cdot sVar(z)}}$

10:32 ничего нового не появляется

10:36 старый текст очищается, появляется новый (это исправленный вариант того, что было в 10:32):

Некоторые источники считают признаком мультиколлинеарности:

* $VIF_j > 10$

* $sCorr(x,z)>0.9$

10:57 фрагмент видео обрывается неожиданно на незаконченной фразе

4-1-2 Что поделать с мультиколлинеарностью?

\url{http://youtu.be/yXQGrWGVpoo}

0:08 убрать слова <<кафедра публичной политики>>  

1:21 Исправить второй пункт на <<Уменьшить дисперсию оценок, пожертвовав их несмещенностью>>

1:28 Исправить третий пункт на <<Мечта: уменьшить дисперсию оценок, используя больше наблюдений>>

2:05 изменяем заголовок слайда на <<Жертвуем несмещнностью, чтобы снизить дисперсию>>

2:10 надпись <<Модель зависит от всех регрессоров>> убираем полностью

2:15 Исправить надпись на 

* Выкинуть часть регрессоров. 

Жертвуем: знанием выкидываемых коэффициентов, несмещенностью оставшихся коэффициентов.

2:23 Исправить надпись на

* Использовать МНК со штрафом

Жертвуем: несмещенностью коэффициентов

2:24 повторную надпись <<Жертвуем несмещенностью!>> убираем полностью


\newpage
4-1-3 изменить название фрагмента на <<Ридж и LASSO регрессия>>

\url{http://youtu.be/fwqs8mPEZFc}

0:08 убрать слова <<кафедра публичной политики>>  

0:23 появляется надпись:

Общий принцип:
\[
\min_{\hat{\beta}} RSS + \text{ Штраф }
\]

0:46 про общий принцип надпись стираем, теперь появляется:

Ридж-регрессия (ridge)
\[
\min_{\hat{\beta}} \sum_{i=1}^n (y_i-\hat{y}_i)^2 + \lambda (\hat{\beta}_1^2 + \hat{\beta}_2^2 + \ldots + \hat{\beta}_k^2)
\]
% \sum_{j=1}^k \hat{\beta}_j^2

0:57 дополнительно появляется: 

LASSO регрессия
\[
\min_{\hat{\beta}} \sum_{i=1}^n (y_i-\hat{y}_i)^2 + \lambda (|\hat{\beta}_1| + |\hat{\beta}_2| + \ldots + |\hat{\beta}_k|)
\]
% \sum_{j=1}^k |\hat{\beta}_j|

1:10 дополнительно появляется:

Метод эластичной сети
\[
\min_{\hat{\beta}} \sum_{i=1}^n (y_i-\hat{y}_i)^2 + \lambda_1 \sum_{j=1}^k |\hat{\beta}_j| + \lambda_2 \sum_{j=1}^k \hat{\beta}_j^2
\]



4-1-4 Идея метода главных компонент

\url{http://youtu.be/6LnUUI0Iotw}

0:08 убрать слова <<кафедра публичной политики>>  

2:29 исправить на:

* переменная $pc_1$ имеет максимальную выборочную дисперсию $sVar(pc_1)$

3:04 исправить на:

* переменная $pc_2$ некоррелирована с $pc_1$ и имеет максимальную $sVar(pc_2)$

3:33 исправить на:

* переменная $pc_3$ некоррелирована с $pc_1$, $pc_2$ и имеет максимальную $sVar(pc_3)$

* и т.д.

4:30 выше фразы <<Первая главная компонента --- математика>> добавить слово <<Упрощённо:>>

5:07 --- 5:14 процесс рисования облака точек можно ускроить



4-1-5 Пример нахождения главной компоненты

\url{http://youtu.be/oRYyXjZJipo}

0:08 убрать слова <<кафедра публичной политики>>  



\newpage
4-1-6 Свойства главных компонент 

\url{http://youtu.be/-UcSI47ITfU}

0:08 убрать слова <<кафедра публичной политики>>  

0:35 исправить вторую формулу (вместо нижнего индекса $k$ должен быть $2$), в результате должно получиться:

$pc_1=v_{11} \cdot x_1 +  v_{21} \cdot x_2 + \ldots + v_{k1} \cdot x_k$

$pc_2=v_{12} \cdot x_1 +  v_{22} \cdot x_2 + \ldots + v_{k2} \cdot x_k$

\ldots

0:42 исправить формулу на

$sCorr(pc_j, pc_m)=0$

0:52 исправить формулу на

\[
sVar(x_1)+ sVar(x_2) + \ldots + sVar(x_k) =
sVar(pc_1)+ sVar(pc_2) + \ldots + sVar(pc_k)
\]

2:13 --- 2:43 --- удалить полностью (эти же слова более аккуратно я говорю с момента 2:43). В надпись, появляющуюся в 2:13 добавить тире, чтобы вышло:

Если: все переменные центрированы, $\bar{x}_j=0$

То: $pc_j=X \cdot v_j$ и $|pc_j|^2=\lambda_j$, 

где $\lambda_j$ --- собственные числа,

$v_{j}$ --- собственные вектора матрицы $X'X$

5:40 исправить второй пункт на 

* Бездумное применение перед регрессией

6:13 исправить знаменатель в формуле в конце текста, должно быть:

$x_j=\frac{a_j-\bar{a}_j}{se(a_j)}$

7:10 появляется подзаголовок

Применение метода главных компонент перед регрессией

7:21 ниже подзаголовка появляется надпись:

Небезопасная процедура:

Шаг 1. Найти главные компоненты, $pc_1$, $pc_2$

7:28 вместо текущего текста появляется 

Шаг 2. Построить регрессию $y$ на $pc_1$, $pc_2$

7:39 вместо текущего текста появляется:

Проблемы:

* коэффициенты при $pc_j$ сложнее интерпретировать

8:03 добавляется строка

* самый изменчивый регрессор $x$ гипотетически может быть наименее связан с $y$

8:43 Вместо <<Мораль --- мультиколлинеарность>> оставляем просто <<Мультиколлинеарность>>

8:46 исправляем пункт на 

* сильная линейная зависимость между регрессорами


4-2-1 Доверительные интервалы при мультиколлинеарности

\url{http://youtu.be/9P4pCJN9ENw}

0:08 убрать слова <<кафедра публичной политики>>  

переснять, т.к. нет \verb|lab_04_before.R|

1:22  показать график


4-2-2 Lasso регрессия 

\url{http://youtu.be/ixdKGZJeSt0}

0:08 убрать слова <<кафедра публичной политики>>  


\newpage
4-2-3 Ридж-регрессия и идея оценки лямбды

\url{http://youtu.be/dAJu-BUyvks}

0:08 убрать слова <<кафедра публичной политики>>  



4-2-4 Метод главных компонент

\url{http://youtu.be/g83D9xxOWzE}

0:08 убрать слова <<кафедра публичной политики>>  



\end{document}